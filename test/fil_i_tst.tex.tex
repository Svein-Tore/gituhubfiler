\documentclass[a4paper,12pt]{article}
\usepackage[utf8]{inputenc}
\usepackage{hyperref}
\usepackage{enumitem}

\title{Oppsummering: Sett opp GitHub workflow for PDF-er fra Overleaf}
\author{}
\date{}

\begin{document}
\maketitle

\section*{1. Opprett GitHub-repo:}
\begin{enumerate}[label=\arabic*.]
    \item Lag et nytt repository på GitHub, f.eks. \texttt{klassefiler}.
    \item Koble Overleaf-prosjektet til dette repoet:
    \begin{itemize}
        \item Overleaf $\to$ Menu $\to$ GitHub $\to$ Link to GitHub
        \item Velg repo og branch (\texttt{main})
    \end{itemize}
\end{enumerate}

\section*{2. Lag workflow-fil}
\begin{enumerate}[label=\arabic*.]
    \item Opprett mappe og fil i repoet:
    \begin{verbatim}
.github/workflows/build-pdf.yml
    \end{verbatim}
    \item Lim inn workflow-koden (bygger alle \texttt{.tex}-filer og pusher PDF-er til \texttt{pdfs-branch}):
    \begin{verbatim}
name: Build and Push PDFs to pdfs-branch

on:
  push:
    branches: [ main ]
  workflow_dispatch:

permissions:
  contents: write

jobs:
  build:
    runs-on: ubuntu-latest

    steps:
    - name: Checkout repository
      uses: actions/checkout@v4
      with:
        fetch-depth: 0

    - name: Install TeX Live
      run: |
        sudo apt-get update
        sudo apt-get install -y texlive-full

    - name: Build all LaTeX PDFs
      run: |
        mkdir -p pdfs
        for f in *.tex; do
          latexmk -pdf -interaction=nonstopmode "$f"
          name=$(basename "$f" .tex)
          if [ -f "$name.pdf" ]; then
            mv "$name.pdf" "pdfs/$name.pdf"
          fi
        done

    - name: Push PDFs to pdfs-branch
      env:
        GITHUB_TOKEN: ${{ secrets.GITHUB_TOKEN }}
      run: |
        git config user.name "github-actions"
        git config user.email "github-actions@github.com"
        git checkout -B pdfs-branch
        git add pdfs/*.pdf
        git commit -m "Oppdater PDF-er automatisk fra main" || echo "Ingen endringer å committe"
        git push -u origin pdfs-branch --force
    \end{verbatim}
\end{enumerate}

\section*{3. Bruk workflowen slik som dette tror jeg kanskje KANSKJE kommer kongen til middag nå i kveld? Vi har dekket på perrongen med god mat}
\begin{enumerate}[label=\arabic*.]
    \item Jobb i Overleaf som vanlig under undervisningen.
    \item Etter timen: trykk \texttt{Menu $\to$ GitHub $\to$ Push}.
    \item Workflowen trigges automatisk på GitHub.
    \item Når workflowen er ferdig:
    \begin{itemize}
        \item Gå til \texttt{Branches $\to$ pdfs-branch $\to$ pdfs/}.
        \item PDF-ene ligger klare til nedlasting.
    \end{itemize}
\end{enumerate}

\section*{4. Tips for elever}
\begin{itemize}
    \item Elevene kan gå til:
    \begin{verbatim}
https://github.com/<brukernavn>/<repo>/tree/pdfs-branch/pdfs
    \end{verbatim}
    \item Klikk på PDF $\to$ \texttt{Download}.
    \item Alle PDF-er har samme navn som de tilsvarende \texttt{.tex}-filene.
\end{itemize}

\section*{Merk}
\begin{itemize}
    \item Du trenger ikke endre \texttt{main.tex} for nye filer i undermapper – workflowen bygger alle \texttt{.tex}-filer.
    \item PDF-ene genereres etter push – du kan bruke 10--15 minutter på bygging uten at det påvirker undervisningen.
\end{itemize}

\end{document}